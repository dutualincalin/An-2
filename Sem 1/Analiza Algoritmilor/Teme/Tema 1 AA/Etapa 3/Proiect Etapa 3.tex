\documentclass[10pt]{report}
\usepackage{graphicx}																		
\usepackage[romanian]{babel}
\usepackage[unicode]{hyperref} 
\usepackage{amsmath,epsfig,pifont,calc,pifont,pstricks}
\usepackage{rom} 						 % pentru a scrie cu diacritice in limba romana	
\usepackage{color}
\usepackage{listings}
\definecolor{mygreen}{RGB}{28,172,0} % color values Red, Green, Blue
\definecolor{mylilas}{RGB}{170,55,241}
\usepackage{placeins}

\usepackage{etoolbox}
\patchcmd{\thebibliography}{\chapter*}{\section*}{}{}

\usepackage{titlesec}
\titleformat{\chapter}
  {\normalfont\fontsize{14}{19}\sffamily\bfseries\color{mylilas}}
  {\thechapter}
  {1em}
  {}

\titleformat{\section}
  {\normalfont\fontsize{10}{17}\sffamily\bfseries\color{mylilas}}
  {\thesection}
  {1em}
  {}

\titleformat{\subsection}
  {\normalfont\fontsize{10}{17}\sffamily\bfseries\color{mylilas}}
  {\thesubsection}
  {1em}
  {}

\titleformat{\subsubsection}
  {\normalfont\fontsize{10}{17}\sffamily\slshape\color{mylilas}}
  {\thesubsubsection}
  {1em}
  {}
	
%%% setari ale paginii
%indentarea la inceput de paragraf
\setlength{\parindent}{3ex}

%dimensiunea textului pe pagina
\setlength{\voffset}{-2cm}
\setlength{\textheight}{23cm}  
\setlength{\textwidth}{16cm}
\setlength{\topmargin}{0cm}
\setlength{\headsep}{1cm}
\renewcommand{\baselinestretch}{1.2}
\newcommand{\myindent}{\hspace*{3ex}}
%\renewcommand{\baselinestretch}{1}

%margini
\setlength{\oddsidemargin}{0.5cm}
\setlength{\evensidemargin}{-0.3cm}
%\raggedright
\raggedbottom

%% begin preambul - macro-uri definite de autor
\newcommand{\D}{\mathrm{d}}	% va fi folosita in mediul matematic, pentru diferentiala d
\newcommand{\I}{\mathrm{i}}	% va fi folosita in mediul matematic, pentru unitatea imaginara
\newcommand{\eul}{\mathrm{e}}	% numarul lui Euler
\newcommand{\vect}[1]{\mathbf{#1}}	% comanda cu un argument, pentru scrierea vectorilor cu lidere aldine, drepte

\begin{document}
\title{\LARGE{\textbf{Proiect Analiza Algoritmilor \\ Structuri de Date - Cozi de prioritate \\ Arbore AVL 'si Max-Heap}}}
\author{Du'tu Alin C'alin\\323 CD\\\\Facultatea de Automatic'a 'si Calculatoare\\Universitatea Politehnic'a Bucure'sti}
\date{\parbox{\linewidth}{\centering%
  \today\endgraf\bigskip
\textit{\textbf{Abstract.} Contextul acestei lucr'ari este prezentarea temei 1 la disciplina Analiza Algoritmilor parcurs'a 'in primul semenstru al anului  universitar 2020 - 2021. Aceast'a lucrare are ca scop analizarea anumitor opera'tii efectuate de anumite structuri de date abstracte. Se iau 'in vedere implementarea, testarea 'si analiza algoritmilor ce se afl'a 'in leg'atur'a cu scopul structurilor de date, dar 'si analiza beneficiilor pe care le aduc ace'sti algoritmi cu scopul g'asirii celei mai bune op'tiuni pentru diverse situa'tii. Raportul con'tine 13 pagini strcturate pe 5 capitole care ofer'a o analiz'a ampl'a a temei abordate.}\\
\textit{\textbf{Cuvinte cheie:} AVL Tree, Max Heap, Element maxim, Complexitate, Timpi de execu'tie}}}
\maketitle
\tableofcontents
\newpage
\chapter{Divide et Impera}
\section{Enun't}
\myindent
Se d'a un pachet de 52 de c'ar'ti de joc. Pachetul nu con'tine Jokeri. Se citesc din fisierul "Pachet carti.in" cele 4 tipuri de c'ar'ti 'intr-o ordine aleatoare urmat'a de num'arul 'si tip-ul unei c'ar'ti din pachet. Toate c'ar'tile vor fi aranjate 'in ordine cresc'atoare 'si 'in ordinea tipurilor a'sa cum este dat'a din fisierul de intrare. As-ul va fi considerat num'arul 11. S'a se g'aseasc'a pozi'tia c'ar'tii citite de la input.
\newline
\newline
\myindent
Pentru aceast'a problem'a avem urm'atoarele nota'tii: ir - inim'a ro'sie; in - inim'a neagr'a; rb - romb; tr - trefl'a.
\vspace{15mm}

\section{Descrierea solu'tiei problemei}
\vspace{10mm}
\myindent
Problema dat'a se poate rezolva folosind un algortim de c'autare. Av\^and avantajul c'a toate c'ar'tile de un tip sunt sortate putem apela la o c'autare binar'a, 'ins'a tipurile nu sunt sortate si nici nu pot fi sortate 'in vreun fel pentru a putea aplica c'autarea binar'a peste tot pachetul, a'sa c'a se va recurge la un compromis.\\
\newline
\myindent
La 'inceputul programului se vor citi datele de intrare 'si se va forma un vector de structuri. Pentru aceast'a metod'a se va folosi structura carte ce con'tine un int pentru num'arul c'ar'tii 'si un string pentru tipul c'ar'tii. Dup'a formarea vectorului se va apela func'tia Find\_Card care va c'auta 'intr-o prim'a etap'a tipul c'ar'tii 'si apoi va c'auta dup'a num'ar. Func'tia prime'ste vectorul, dar 'si indicii de 'inceput 'si sf\^ar'sit ai vectorului, verific'a mijlocul vectorului 'si 'imparte problema 'in dou'a sau 'imparte pachetul 'in dou'a p'ar'ti cu caracteristici diferite. Odat'a cu g'asirea tipului, algoritmul ajunge la eficien'ta cea mai bun'a 'intruc\^at aceast'a bucat'a de vector este ordonat'a cresc'ator. Aplic\^andu-se algoritmul de c'autare binar'a ajungem rapid la cartea c'autat'a 'si 'in final se va afi'sa pozi'tia g'asit'a.

\vspace{10mm}
\section{Prezentarea algoritmului de rezolvare a problemei}

\begin{lstlisting}[language=Python]
structura{
	numar;
	tip;
} carte

int Find_card(pachet, carte, start, end)
	daca start > finish
		daca pachet[start] == carte
			returneaza start;
	
		altfel returneaza -1;

	mid = (start + finish) / 2;
	
	daca pachet[mid] == carte
		returneaza mid;

	daca tipul lui carte == tip pachet[start] si tip pachet[finish]
		daca numar carte < numar pachet[mid]
			returneaza Find_card(pachet, carte, start, mid-1);

		returneaza Find_card(pachet, carte, mid + 1, finish);

	left = Find_card(pachet, carte, start, mid - 1);
	right = Find_card(pachet, carte, mid + 1, finish);

	daca left diferit de -1
		returneaza left;

	returneaza right;


int main()
	citeste n
	pentru i de la 1 la n
		citeste type 
		
		pentru j de la 0 la 13
			pachet[j + i * 13] = carte(j + 2, type);

	citeste searched;
	scrie Find_Card(pachet, searched, 0 , 51) + 1;
	returneaza 0;
\end{lstlisting}

\vspace{10mm}
\section{Aprecierea complexit'a'tii algoritmului propus}
\myindent
Complexitatea algoritmului dat pentru un singur pachet de 52 de c'ar'ti este O(1) deoarece avem un num'ar fix de c'ar'ti, deci un num'ar fix de tipuri de c'ar'ti, astfel vom avea un num'ar finit de intruc'tiuni executate. Pentru mai multe pachete de c'ar'ti puse toate 'intr-un singur loc algoritmul va avea complexitatea O(n) deoarece c\^and se intr'a 'in Find\_cards 'int\^ai se va aplica c'autarea binar'a pe tot pachetul p\^an'a c\^and se va ajunge la un set cu tipul respectiv, ceea ce va scoate o complexitate O(n), iar c'autarea pe un singur set va scoate O(1) deoarece setul are un num'ar fix de c'ar'ti. Dac'a set-urile nu ar avea valori fixe algortimul ar scoate o complexitate O(log m) unde m este num'arul de c'ar'ti dintr-un set deoarece s-ar fac face c'autare binar'a pe un 'sir cresc'ator. Deci, complexitatea 'in cel mai r'au caz este O(n).

\vspace{5mm}
\section{Analiza succint'a asupra eficien'tei algoritmului propus}
\myindent
Acest algoritm, desi este eficient din punct de vedere al timpului de execu'tie, complexitatea spa'tial'a ar fi O(n) deoarece s-ar genera c\^ate dou'a apeluri de func'tii pentru fiecare tip de carte p\^an'a la g'asirea tipului c'ar'tii. Modalitatea prin care s-ar putea sc'apa de aceast'a complexitate este folosirea unei formule matematice, dar asta ar 'insemna s'a nu folosim metoda Divide et Impera. Deci, algoritmul folosit pentru rezolvarea acestei probleme este optim 'in m'asura 'in care se cere 'in cerin't'a rezolvarea doar cu Divide et Impera.

\vspace{10mm}
\section{Exemplificarea aplic'arii algoritmului propus pentru un exemplu sugestiv}
\subsection{Exemplu de input}
\begin{verbatim}
Pachet carti.in
4
ir
in
rb
tr
8 tr

Rezultatul: 46
\end{verbatim}

\subsection{Aplicarea algoritmului}
\begin{figure}[H]
\centering
\includegraphics[scale = 0.17]{Divide et Impera/Exemplu}
\end{figure}
\vspace{10mm}
Pasul 0: Se cite'ste din fi'sierul de intrare n 'si cele n tipuri de c'ar'ti, se creeaz'a vectorul pachet 'si se citesc datele c'ar'tii c'autate (8 de trefl'a). Se apeleaz'a func'tia Find\_Card pe tot pachetul.\\
\newline
Pasul 1.1: Se verific'a dac'a mijlocul pachetului este cartea c'autat'a 'si 'in acest caz nu este. Tipul c'ar'tii de la indicele start nu este trefl'a, deci se va apela de dou'a ori Find\_Card, odat'a pe c'ar'tile de la pozi'tiile 0-24, apel care va returna -1 pentru c'a acea bucat'a nu con'tine 8-ul de trefl'a si 'inc'a odata pe c'ar'tile de la 26 la 37.\\
\newline
Pasul 2.1: Se verific'a mijlocul pachetului 'si observam ca din nou nu se potrive'ste, se mai observ'a c'a tipul c'ar'tii de la indicele start nu este trefl'a, deci se apeleaz'a din nou Find\_Card de 2 ori. Primul apel va returna -1 deoarece toate c'ar'tile acoperite de func'tie, adic'a 26-37 sunt de romb. Al doilea apel va acoperi toate c'artile de trefl'a, adic'a 39-51.\\
\newline
Pasul 3: Se verific'a mijlocul 'si se constat'a o potrivire 'intre cartea c'autat'a 'si cartea de la pozi'tia 45. Se va returna 45.\\
\newline
Pasul 2.2: Apelul din variabila left nu con'tine solu'tia, deci va returnat -1 care condi'tioneaz'a programul s'a returneze 45.\\
\newline
Pasul 1.2: Partea st\^ang'a a pachetului nu con'tine cartea c'autat'a ceea ce 'inseamn'a c'a partea dreapt'a o con'tine. Se returneaz'a 45.\\
\newline
Se va scrie 'in output rezultatul func'tiei Find\_Card + 1, adic'a 46 pentru c'a num'ar'atoarea in program a 'inceput de la 0 'si 'in realitate ea 'incepe de la 1.
\chapter{Greedy}
\section{Enun't}
\myindent
Se citesc 3 numere naturale S, n 'si e cu urm'atoarele semnifica'tii: S este o sum'a de bani care trebuie pl'atit'a folosind bancnote care au valori puterile lui e de la 1 la $e^{n}$. S'a se afi'seze la final num'arul de bancnote folosite. Datele se vor citi din fi'sierul Euro.in.\\

Sursa problemei -$>$ \cite{Greedy}
\vspace{5mm}
\section{Descrierea solu'tiei problemei}
\myindent
Pentru rezolvarea acestei probleme se poate opta pentru o variant'a scurt'a 'si simpl'a. Se 'incearc'a aflarea valorii maxime a unei bancnote printr-o iterare de la 0 p\^an'a la puterea maxim'a. Dup'a aflarea valorii maxime va trebui s'a se aplice un concept Greedy foarte simplu pentru minimizarea num'arului de bancnote. Se pl'ate'ste mai 'int\^ai c\^at se poate cu bancnote de valoare maxim'a, apoi c\^and nu se mai poate, se va pl'ati restul sumei cu bancnote a c'aror valoare e din ce 'in ce mai mic'a p\^an'a c\^and suma este pl'atit'a. Se poate afla foarte usor num'arul de bancnote printr-o 'imp'artire a sumei la valoarea maxim'a, num'arul fiind dat de c\^at. Restul va fi pl'atit cu bancnote mai mici. Dup'a ce s-a efectuat plata se va afi'sa rezultatul.

\newpage
\section{Prezentarea algoritmului de rezolvare a problemei}
\begin{lstlisting}[language=Python]
int main()
	citire sum;
	citire max_pow;
	citire baza;

	max = 1;
	nr = 0;

	pentru i de la 0 la max_pow - 1
		daca(max * baza > sum)
			inchide for-ul;

		max = max * baza;

	cat timp sum > 0
		nr = nr + sum / max;
		sum = sum % max;
		max = max / baza;

	scrie nr;
\end{lstlisting}

\vspace{10mm}
\section{Aprecierea complexit'a'tii algoritmului propus}
\myindent
Din punct de vedere temporal, programul con'tine un for de la 0 la max\_pow care va face 'in cel mai r'au caz max\_pow pasi, deci O(max\_pow) 'si un while care depinde de sum'a. Num'arul de itera'tii va depinde de puterea bazei la valoarea maxim'a pe care o poate avea o bancnot'a, adic'a va depinde de num'arul de tipuri de bancnote pe care le avem la dispozi'tie, deci avem complexitate O($log_{baza}max$). Astfel complexitatea final'a va fi O(max\_pow + $log_{baza}max$.

\section{Analiz'a succint'a asupra posibilit'a'tii de ob'tinere a optimului global}
\myindent
Algoritmul respect'a proprietatea de substructur'a optimal'a deoarece solu'tia subproblemei, adic'a num'arul de bancnote pentru un tip ajut'a al rezolvarea problemei, adic'a num'arul total de bancnote. De asemenea, algoritmul respect'a si proprietatea de alegere de tip Greedy deoarece nu se calculeaz'a num'arul bancnotelor pe baza valorilor anterior calculate.

\section{Exemplificarea aplic'arii algoritmului propus pentru un exemplu sugestiv}
\subsection{Exemplu de input}
\begin{verbatim}
Euro.in
444 5 2

Rezultat: 16
\end{verbatim}

\subsection{Aplicarea algoritmului}

\begin{figure}[H]
\centering
\includegraphics[scale = 0.12]{Greedy/for}
\end{figure}

\vspace{5mm}
\myindent
Algoritmul 'incepe prin citirea parametrilor 'si intrarea 'in for care stabile'ste valoarea maxim'a a unei bancnote 'in func'tie de puterea maxim'a 'si 'in func'tie de sum'a. Practic, fiecare itera'tie din for va verifica urm'atoarea putere a bazei, astfel va ie'si puterea cea mai mare.

\begin{figure}[H]
\centering
\includegraphics[scale = 0.2]{Greedy/while}
\end{figure}
\myindent
Dup'a g'asirea maximului se va itera prin while-ul care va scoate num'arul de bancnote. Aici se va 'imp'ar'ti suma la valoarea maxim'a 'si va folosi: c\^atul ca s'a numere bancnotele folosite 'si restul c\^atului pentru restul de plat'a. La urm'atoarea itera'tie se va num'ara c\^at din restul r'amas se poate pl'ati cu bancnote cu urm'atoarea valoare maxim'a si tot a'sa. 'In final, dup'a ce se pl'ate'ste toat'a suma, while-ul se 'incheie 'si se afi'seaz'a num'arul de bancnote.
\chapter{Programare Dinamic'a}
\section{Enun't}
\vspace{5mm}
\myindent
Pentru a prepara o ciorb'a bun'a pentru so'tul ei, Rodica se duce la pia't'a pentru a cump'ara ingredientele necesare. Ea are o list'a "Ingrediente.in" cu num'arul v\^anz'atorilor de legume disponibili 'si cu suma pe care o de'tine pe prima linie. Pe urm'atoarele n linii sunt ofertele fiec'arui v\^anz'ator sub forma unei perechi de tip (kg, pre't). Rodica dore'ste s'a afle folosind tehnica Program'arii Dinamice care este num'arul maxim de kilograme pe care le poate cump'ara cu to'ti banii pe care 'ii are.

\vspace{10mm}
\section{Descrierea solu'tiei probemei}
\myindent
Pentru a utiliza tehnica Program'arii dinamice, 'in aceast'a problem'a, trebuie s'a stabilim dimensiunile tabelei. Astfel liniile vor fi reprezentate de v\^anz'atori deoarece cu aceste "obiecte" vom lucra, iar coloanele vor fi reprezentate de suma cheltuit'a pentru c'a avem la dispozi'tie o sum'a maxim'a, o restric'tie. Cu acest tabel putem lua toate posibilit'a'tile 'si a'sa vom g'asi solu'tia cea mai bun'a. Astfel, 'in fiecare c\^amp al tabelei vom verifica care din cele 3 cazuri va fi cel mai bun pentru acea situa'tie. Primul caz este c\^and solu'tia precedent'a este cea mai bun'a 'si atunci vom prelua acea solu'tie. Al doilea caz este c\^and oferta v\^anz'atorului este mai rentabil'a dec\^at solu'tia precedent'a, iar al 3-lea caz este o combina'tie 'intre cele dou'a. 'In final, solu'tia se va reg'asi pe c\^ampul de pe ultima linie 'si ultima coloan'a.

\vspace{10mm}
\section{Prezentarea algoritmului de rezolvare a problemei}
\begin{lstlisting}[language=Python]
structura{
	greutate;
	pret;
} Ingredient;

int main()
	citeste n;
	citeste S;

	pentru i de la 1 la n
		citeste vanzator[i].greutate;
		citeste vanzator[i].pret;

	pentru i de la 0 la n
		pentru j de la 0 la S
			daca i = 0 sau j = 0
				continua;

			daca j < vanzator[i].pret
				dp[i][j] = dp[i - 1][j];
				continua;

			daca dp[i - 1][j] > vanzator[i].greutate
				dp[i][j] = dp[i - 1][j];

			altfel dp[i][j] = vanzator[i].greutate;

			daca(dp[i][j] < dp[i-1][j - vanzator[i].pret] 
				+ vanzator[i].greutate)

				dp[i][j] =  dp[i-1][j - vanzator[i].pret]
						+ vanzator[i].greutate;

	scrie dp[n][S];
\end{lstlisting}

\vspace{10mm}
\section{Aprecierea complexit'a'tii algoritmului propus}
\myindent
Din punct de vedere al complexit'a'tii temporale algoritmul va executa n *  S pa'si deoarece avem citirile care au O(1), avem citirea vectorului de v\^anz'atori care este intr-un for de la 1 la n, ceea ce inseamn'a O(n), avem apoi iterarea prin tabela dinamic'a care con'tine dou'a for-uri, unul de la 0 la n 'si unul de la 0 la S 'si 'in interiorul lor instructiuni care au complexitate O(1), deci avem O(S*n). Astfel vom avea O(S * n + n) = O(n * (S + 1)) = O(n * S).\\\\
\myindent
'In program exist'a variabile care au o complexitate spa'tiala de O(1), un vector care con'tine structuri cu 2 c\^ampuri care are complexitate O(2 * n) 'si o tabel'a de dimensiuni n+1 'si S+1 care va avea o complexitate spa'tiala O((n + 1) *  (S + 1)) = O(n * S). Apeluri recursive nu exist'a 'in acest program, deci complexitatea spa'tial'a r'am\^ane O(n * 2) pentru S $<$= 2 'si O(n * S) pentru S $>$ 2.\\

\vspace{10mm}
\section{Explicarea modului 'in care a fost ob'tinut'a rela'tia de recuren't'a}
\myindent
'In principiu, dac'a la linia i avem v\^anz'atorul cu oferta (kg, pret) vom avea pe aceast'a linie de la prima coloan'a p\^an'a la coloana pret solu'tia precedent'a, adic'a dp[i - 1][nr. coloan'a]. Apoi avem de verificat care din cele 3 solu'tii este cea mai bun'a. Se va compara solu'tia de la linia precedent'a cu oferta v\^anz'atorului actual 'si se va obtine o valoare maxim'a. Se verific'a apoi dac'a este mai rentabil'a cumularea greut'a'tii ofertei actuale cu solu'tia de la linia trecut'a dar sc'az\^and din coloana actual'a costul actualului v\^anz'ator.\\

\vspace{10mm}
\section{Exemplificarea aplic'arii algoritmului propus pentru un exemplu sugestiv}
\subsection{Exemplu de input}
\begin{verbatim}
Ingrediente.in
5 30
3 9
1 3
2 5
5 15
4 10

Rezultat: 11
\end{verbatim}
\newpage
\subsection{Aplicarea algoritmului}
\begin{figure}[H]
\centering
\includegraphics[scale = 0.7]{Programare Dinamica/Tabel}
\caption{Tabelul PD}
\end{figure}
\vspace{10mm}
\myindent
Dup'a citirea 'si ini'tializarea tuturor vectorilor urmeaz'a construirea solu'tiei folosind tabela PD. 'Stiind c'a tabela este ini'tializat'a la 0 nu avem nimic de modificat la linia 'si coloana cu indicele 0.\\
\newline
\myindent
Astfel, pe prima linie avem p\^an'a la costul ofertei adic'a coloana 9 solu'tia precedent'a 'si apoi vom avea doar 3 kilograme p\^an'a la final deoarece nu sunt alte oferte care s'a o precede pe aceasta. La a doua linie, la fel, p\^ana la coloana 3 solu'tia precedent'a. De la coloana 3 la 8 oferta actual'a este cea mai bun'a, deci se va trece 1 kilogram, de la 9 la 11 solu'tia precedent'a este cea mai bun'a, iar de la 12 p\^an'a la final cumularea primelor dou'a oferte va fi cea mai optim'a op'tiune.\\
\newline
\myindent
Linia a 3-a va avea p\^an'a la coloana a 4-a solu'tia precedent'a, coloanele 5 - 7 vor avea oferta actual'a, iar de la 8 'incolo se vor face cumul'ari de solu'tii. Linia a 4-a de la 0 la 19 se iau solu'tiile precedente 'si de la 20 'incolo se vor cumula ofertele, iar la linia 5 de la 0 la 9 se va lua solu'tia precedent'a, coloanele 10, 11 'si 12 vor lua oferta actual'a 'si de la 13 p\^an'a la finalul liniei se vor face cumul'ari de oferte.\\
\newline
\myindent
'In final putem extrage solu'tia de la ultima linie 'si ultima coloan'a a tabelei dp.
\chapter{Backtracking}
\section{Enun't}
\myindent
Se cite'ste din fisierul plata.in care are pe prima linie num'arul tipurilor de bancnote urmat de o sum'a, pe a doua linie valorile bancnotelor, iar pe a treia linie num'arul de bancnote disponibile pentru fiecare valoare citit'a de la linia 2. S'a se genereze modalit'a'tile de plat'a sub form'a de submul'timi a sumei cu bancnotele disponibile. Submul'timile trebuie s'a fie 'in ordine cresc'atoare 'si s'a nu se repete.

\vspace{10mm}
\section{Descrierea solu'tiei probemei}
\myindent
Pentru a genera toate combina'tiile posibile, dar av\^and anumite restric'tii se poate apela f'ar'a ezit'ari la Backtracking. Algoritmul are 3 func'tii. Prima este main 'in care se citesc toate datele 'si se creaz'a vectorul de bancnote 'in care se va insera c\^ate o bancnot'a. A doua este backtracking care creeaz'a un vector de solu'tii, se itereaz'a prin vectorul de domeniu sau vectorul de bancnote 'si se verific'a dac'a valoarea din domeniu e mai mare dec\^at ultima valoare din solu'tie (dac'a valoarea exist'a). Dac'a este, atunci se adaug'a acel element 'in solu'tie, 'il scoate din domeniu 'si apoi prin recursivitate se creeaz'a toate combina'tiile posibile care au acela'si 'inceput. Odat'a ce s-a ajuns la o solu'tie se verific'a dac'a este valid'a, adic'a dac'a suma solu'tiei este exact suma c'autat'a. Dac'a da, aici intervine partea cu restric'tii. Se va verifica dac'a solution se reg'ase'ste deja printre solu'tiile g'asite utiliz\^and a treia func'tie, is\_not\_in\_sols. Dac'a solu'tia e nou'a atunci ea va fi ad'augat'a la vectorul de solu'tii. 'In final se va afi'sa vectorul de solu'tii.

\section{Prezentarea algoritmului de rezolvare a problemei}
\begin{lstlisting}[language=Python]
int is_not_in_sols(solution, sols)
	pentru i de la 0  la marimea lui sols - 1
		daca (marimea lui sols(i) = marimea lui solution)
			same = 1;

			pentru j  de la 0  la marimea lui sols(i) - 1
				daca (sols(i)(j) diferit de solution(j))
					same = 0;
					iesi din for

			daca (same = 1)
				returneaza 0;

	returneaza 1;


void bactracking(domain, solution, sols, sol_sum, sum)
	daca (sol_sum > sum)
		iesi din functie;

	daca (sol_sum = sum)
		daca(is_not_in_sols(solution, sols)  = 1)
			adauga solution in sols;
			iesi din functie;

	pentru i de la 0 la marimea lui domain - 1
		daca solution are elemente 
		   si domain(i) < ultimul element al lui solution
			continua;

		new_domain = domain;
		scoate domain(i) din new_domain;

		new_solution = solution;
	adauga domain(i) in solution;

	new_sol_sum = sol_sum + domain(i);

	backtracking(new_domain, new_solution, sols, new_sol_sum, sum);


int main()
	citeste n;
	citeste S;
	
	citeste types;
	
	pentru i de la 0 la n - 1
		citeste times;
		pentru j de la 0 la times -1
			adauga types(i) in domain;

	backtracking(domain, solution, sols, 0, S);

	pentru i de la 0 la marimea lui sols
		pentru j de la 0 la marimea lui sols(i)
			scrie sols(i)(j);

		scrie pe urmatoarea linie

	returneaza 0;
\end{lstlisting}

\vspace{10mm}
\section{Aprecierea complexit'a'tii algoritmului propus}
\myindent
Din punct de vedere temporal algoritmul con'tine un for la citire de complexitate O(n * max(times)) 'si un for pentru scriere de complexitate O(sol.size * sols(i).size). 'In func'tia is\_not\_in\_sols algoritmul itereaz'a prin toate solutiile lui sols deci avem complexitate O(sol.size * sols(i).size), iar 'in func'tia backtracking, pentru c'a avem un for care trece prin tot domeniul 'si un apel recursiv care va genera toate combina'tiile posibile, vom avea complexitate O(m *  m!) unde m este num'arul de bancnote. Compar\^and cele 3 complexit'a'ti pe cazul cel mai r'au ar rezulta O(n * max(times)) $<$ O(sol.size * sols(i).size) $\le$  O(m * m!) pentru c'a generarea solu'tiilor ia cel mai mult timp 'si pentru c'a num'arul solu'tiilor va fi evident mai mic dec\^at num'arul itera'tiilor de backtracking. 'In concluzie complexitatea temporal'a este O(m *  m!), unde m este num'arul de bancnote disponibile.\\
\newline
Din punct de vedere spa'tial, programul are un vector care stocheaz'a tipurile de bancnote, deci O(n) 'si un vector domain care stocheaz'a toate bancnotele, una c\^ate una, deci O(m), unde m este num'arul de bancnote disponibile.\\
\newline
Astfel pentru acest program avem complexitate temporal'a O(m * m!) 'si complexitate spa'tial'a O(m), unde m este num'arul de bancnote. 

\vspace{10mm}
\section{Analiz'a succint'a asupra eficien'tei algoritmului propus}
\myindent
Algoritmul nu are foarte multe optimiz'ari. Singura optimizare este 'in momentul 'in care se alege din domeniu o valoare 'si se impune restric'tia pentru multimi cresc'atoare. Dac'a elementul din domeniu e mai mic dec\^at ultimul element din solu'tie, atunci se trece la urm'atoarea itera'tie, acest lucru elimin\^and o bun'a parte din combina'tii. In plus se mai pot aplica optimiz'arile din compilator cu O1, O2 sau O3, dar altfel consider acest algoritm ca fiind eficient pentru aceast'a problem'a.

\vspace{10mm}
\section{Exemplificarea aplicarii algoritmului propus pentru un exemplu sugestiv}
\subsection{Exemplu de input}
\begin{verbatim}
plata.in
4 40
1 5 10 50
5 3 3 1

Rezultat:
1 1 1 1 1 5 5 5 10 10
1 1 1 1 1 5 10 10 10
5 5 10 10 10
\end{verbatim}
\subsection{Aplicarea algoritmului}

Algoritmul va primi datele de intrarea si va forma vectorul domain cu toate bancnotele disponibile si apoi se va apela func'tia backtracking. Acesta va fi parcursul:\\\\
\myindent
Pasul 1:   Solution = \{1\}; Domain = \{1, 1, 1, 1, 5, 5, 5, 10, 10, 10, 50\}\\
\myindent
Pasul 2:   Solution = \{1, 1\}; Domain = \{1, 1, 1, 5, 5, 5, 10, 10, 10, 50\}\\
\myindent
Pasul 3:   Solution = \{1, 1, 1\}; Domain = \{1, 1, 5, 5, 5, 10, 10, 10, 50\}\\
\myindent
Pasul 4:   Solution = \{1, 1, 1, 1\}; Domain = \{1, 5, 5, 5, 10, 10, 10, 50\}\\
\myindent
Pasul 5:   Solution = \{1, 1, 1, 1, 1\}; Domain = \{5, 5, 5, 10, 10, 10, 50\}\\
\myindent
Pasul 6:   Solution = \{1, 1, 1, 1, 1, 5\}; Domain = \{5, 5, 10, 10, 10, 50\}\\
\myindent
Pasul 7:   Solution = \{1, 1, 1, 1, 1, 5, 5\}; Domain = \{5, 10, 10, 10, 50\}\\
\myindent
Pasul 8:   Solution = \{1, 1, 1, 1, 1, 5, 5, 5\}; Domain = \{10, 10, 10, 50\}\\
\myindent
Pasul 9:   Solution = \{1, 1, 1, 1, 1, 5, 5, 5, 10\}; Domain = \{10, 10, 50\}\\
\myindent
Pasul 10: Solution = \{1, 1, 1, 1, 1, 5, 5, 5, 10, 10\}; Domain = \{10, 50\}\\
\myindent
Pasul 11: Solution = \{1, 1, 1, 1, 1, 5, 5, 5, 10, 10, 10\}; Domain = \{50\}\\
\myindent
-$>$ suma bancnotelor din Solution este 40, deci se adaug'a Solution in sols.\\

Urmeaz'a un set de pasi unde solu'tiile vor fi identice cu prima solu'tie p\^an'a c\^and:\\

Pasul 12: Solution = \{1, 1, 1, 1, 1, 5\}; Domain = \{ 10, 10, 10, 50\}\\
\myindent
Pasul 13: Solution = \{1, 1, 1, 1, 1, 5, 10\}; Domain = \{10, 10, 50\}\\
\myindent
Pasul 14: Solution = \{1, 1, 1, 1, 1, 5, 10, 10\}; Domain = \{10, 50\}\\
\myindent
Pasul 15: Solution = \{1, 1, 1, 1, 1, 5, 10, 10, 10\}; Domain = \{50\}\\
\myindent
-$>$ suma bancnotelor din Solution este 40, deci se adaug'a Solution in sols.\\

Urmeaz'a din nou un set de solu'tii care se vor repeta p\^an'a c\^and:\\

Pasul 16: Solution = \{5\}; Domain = \{5, 10, 10, 10, 50\}\\
\myindent
Pasul 17: Solution = \{5, 5\}; Domain = \{10, 10, 10, 50\}\\
\myindent
Pasul 18: Solution = \{5, 5, 10\}; Domain = \{10, 10, 50\}\\
\myindent
Pasul 19: Solution = \{5, 5, 10, 10\}; Domain = \{10, 50\}\\
\myindent
Pasul 20: Solution = \{5, 5, 10, 10, 10\}; Domain = \{50\}\\
\myindent
-$>$ suma bancnotelor din Solution este 40, deci se adaug'a Solution in sols.\\

P\^an'a la finalul execu'tiei metoda backtracking nu va mai g'asi alte combina'tii. 'In final se vor afi'sa 'in output solu'tiile:\\
\myindent
1 1 1 1 1 5 5 5 10 10\\
\myindent
1 1 1 1 1 5 10 10 10\\
\myindent
5 5 10 10 10\\
\myindent




\chapter{Analiz'a comparativ'a}
\section{Divide et Impera}
\myindent
Tehnica Divide et Impera aduce multe avantaje mai ales la sort'ari deoarece cu ajutorul acestei tehnici s-a ajuns s'a se creeze algoritmi de sortare foarte eficien'ti cum ar fi Merge Sort care are cea mai bun'a complexitate pe toate cele 3 cazuri dintre to'ti algoritmii de sortare, O(n * log n). De asemenea, aceast'a metod'a a introdus 'si algoritmul de c'autare binar'a care efectuat pe un vector sortat va avea o complexitate O(log n) care este printre cele mai bune din toat'a lista de algoritmi de c'autare, 'ins'a dac'a nu este sortat, atunci c'autarea binar'a va avea O(n) care este asem'an'atoare cu complexitatea c'aut'arii liniare. De asemenea, tot 'in acest caz este mai pu'tin eficient 'a din punct de vedere al timpului de execu'tie c'autarea binar'a dac'a se apeleaz'a 'impreun'a cu o func'tie de sortare fa't'a de al'te metode cum ar fi chiar c'autarea liniar'a. Astfel, Divide et Impera este o tehnic'a eficient'a, 'ins'a doar pentru anumite probleme 'si 'in anumite condi'tii.

\vspace{10mm}
\section{Greedy}
\myindent
Tehnica Greedy 'imbin'a eficien'ta cu simplitatea rezolv'arii unui algoritm folosind "scurt'aturi" pentru a construi c\^at mai rapid solu'tiile. De exemplu, 'in Problema Rucsacului unde avem un num'ar de obiecte care au o greutate 'si o valoare vom aplica tehnica Greedy prin compararea obiectelor cu rezultatul c\^atului dintre pre'tul si greutatea obiectelor. Un avantaj este c'a vom ob'tine o solu'tie mult mai rapid'a 'si mai eficent'a, 'ins'a nu este 'si cea mai corect'a solu'tie. Pe anumite input-uri acest ra'tionament nu scoate solu'tia cea mai optim'a ceea ce este un dezavantaj pentru tehnica Greedy. Pentru a rezolva acest neajuns putem folosi Programare Dinamic'a.

\vspace{10mm}
\section{Programare dinamic'a}
\myindent
Programarea dinamic'a are la baz'a un concept cu totul diferit fa't'a de Greedy, construie'ste solu'tia actual'a dup'a solu'tiile precedente folosind un vector sau o matrice unde sunt re'tinute toate solu'tiile. Avantajul fa't'a de Greedy este c'a solu'tia creat'a va fi mereu cea mai optim'a, 'ins'a un mare dezavantaj 'il constituie crearea vectorului/matricei de solu'tii care pentru input-uri mari au o dimensiune foarte mare consum\^and astfel timp 'si resurse. Pe de alt'a parte exist'a algoritmi eficien'ti care au la baz'a programare dinamic'a. Dac'a avem o recuren'ta, de exemplu, 'in loc s'a folosim un vector 'in care s'a re'tinem solu'tiile, putem folosi tehnica de exponen'tiere pe matrice pentru recuren'te liniare care o complexitate excelent'a, O(log n) care scoate un timp excelent, mult mai bun dec\^at solu'tia clasic'a. Astfel, Programarea Dinamic'a este 'in general folosit'a eficient la problemele care se rezolv'a cu recuren'te sau cu solu'tii care depind de cele precedente.

\vspace{10mm}
\section{Backtracking}
\myindent
Metoda Backtracking este folosit'a pentru a crea toate solu'tiile posibile av\^and anumite restric'tii de care s'a putem 'tine cont. Este folosit 'in general 'in problemele unde trebuiesc generate mul'timi sau submul'timi pe baza unui domeniu dat 'si care s'a respecte restric'tiile impuse. Avantajul acestei metode este eficien'ta 'in compara'tie cu metoda brute-force care genereaz'a toate posibilit'a'tile si deabia dup'a le selecteaz'a, 'ins'a chiar 'si a'sa complexitatea algoritmilor care au la baz'a Backtracking este foarte mare. Fa't'a de celelalte metode Backtraking este mai mult ca o ultim'a alternativ'a, ceva ce se poate folosi eficient doar dac'a nu exist'a alte posibilit'a'ti 'si din acest punct de vedere se poate considera o metod'a eficient'a.

\chapter{Concluzii}
'In concluzie, pentru situa'tiile 'in care este nevoie de o baz'a de date nu foarte populat'a cu pu'tine ad'aug'ari 'si 'stergeri de date se poate opta pentru o structur'a de date de tip AVL, iar pentru cazul 'in care este nevoie de o baz'a de date al c'aror principal atribut este un num'ar prioritar se poate folosi pentru cea mai bun'a performan't'a un Max - Heap.

\vspace{5 mm}
\nocite{*}
\bibliographystyle{plainrom}
\addcontentsline{toc}{chapter}{Bibliografie}
\bibliography{Fisiere/Bibliografie}

\end{document}
