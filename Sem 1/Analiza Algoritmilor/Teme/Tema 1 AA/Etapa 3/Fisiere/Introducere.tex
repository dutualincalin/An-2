\chapter{Introducere}
\section{Descrierea problemei rezolvate 'si specificarea solu'tiilor alese}
Problema aleas'a este cea a analizei opera'tiilor unor structuri de date abstracte. Structurile alese fac parte din categoria cozilor de prioritate, astfel s-au ales pentru compara'tie arborele AVL 'si Max-Heap-ul 'si se vor compara prin utilizarea opera'tiilor comune acestora.
\newline 
\myindent
AVL tree este un arbore binar de c'autare care are proprietatea de auto-echilibare ce 'ii permite s'a execute opera'tiile unui arbore binar de c'autare obi'snuit mult mai eficient. Pentru AVL se iau 'in vedere opera'tiile de creare a structurii, ad'augare a unui element, opera'tiile de rota'tie st'anga 'si rota'tie dreapta, g'asirea elementului maxim 'si minim, eliminarea elementului maxim 'si a celui minim din structur'a 'si 'stergerea structurii. Max-Heap-ul este o structur'a de date bazat'a pe un arbore complet cu proprietatea c'a nodul cu valoarea maxim'a se gase'ste chiar 'in r'ad'acin'a. Pentru Max - Heap se vor implementa opera'tiile de creare a structurii, ad'augare a unui element, g'asirea elementului maxim, swift Up si swift Down, extragerea elementului maxim 'si 'stergerea structurii.
\vspace{5 mm}
\section{Specificarea criteriilor de evaluare}
Pentru evaluarea structurilor se vor folosi teste variate, cresc'atoare, descresc'atoare 'si generate random folosind un instrument online de generare a numerelor in ordine aleatoare\cite{Random} at'at pentru AVL c'at 'si pentru Max - Heap. Principalele criterii de evaluare luate 'in calcul pentru ambele structuri sunt timpii de execu'tie care vor testa rapiditatea structurilor pentru diferite input-uri, calculul complexit'a'tilor 'si dificulatatea implement'arii structurii propriu-zise.
\section{Exemple de aplica'tii practice pentru problema aleas'a}
Av\^and 'in vedere propriet'a'tile AVL-ului fa't'a de alte structuri cum ar fi Arborii Ro'su - Negru se poate constata c'a AVL-ul nu mai este foarte folosit in practic'a deoarece exist'a structuri mult mai eficiente, 'insa un exemplu care ar fi perfect pentru folosirea acestei structuri este gestionarea datelor pentru inventarul trenurilor dintr-o re'tea feroviar'a deoarece exemplul se bazeaz'a pe necesitatea c'aut'arii unui tren 'in func'tie de greutate sau de inventar 'si nu necesit'a foarte multe inser'ari 'si 'stergeri de obiecte. Max heap-ul are avantajul de a avea nodul cel mai mare 'in r'ad'acin'a 'si de aceea am putea avea ca exemplu de aplica'tie gestionarea pacien'tior dintr-un centru de analize 'in functie de prioritatea lor 'intruc\^at se poate afla foarte repede care va fi urm'atorul pacient. Aceste dou'a exemple au fost implementate suplimentar pe l\^ang'a testele pentru opera'tii 'si vor oferi informa'tii dintr-o prespectiv'a practic'a care vor contribui la compararea structurilor.